\section{Изпълнимост}

\begin{itemize}
\item
  Релация на еквивалентност = рефлексивност + транзитивност + антисиметричност.
\item
  Частична наредба
\item
  Линейни наредби
  \begin{itemize}
  \item
    Антирефлексивност (строга линейна наредба):
    \[(\forall x)[\neg P(x,x)].\]
  \item
    Транзитивност:
    \[(\forall x)(\forall y)(\forall z)[P(x,y) \land P(y,z) \to P(x,z)].\]
  \item
    Тоталност (всеки два елемента са сравними):
    \[(\forall x)(\forall y)[x = y \lor P(x,y) \lor P(y,x)].\]
  \item
    Гъстота:
    \[(\forall x)(\forall y)[P(x,y) \to (\exists z)[P(x,z) \land P(z,y)]].\]
  \item
    Наследник:
    \[(\forall x)(\exists y)[P(x,y) \land (\forall z)[P(x,z) \to (y = z \lor P(y,z)]].\]
  \item
    Първи елемент:
    \[(\exists x)(\forall y)[x = y \lor P(x,y)].\]
  \end{itemize}
\end{itemize}

\begin{example}
  Да разгледаме $\L = \{P\}$, където $\sharp P = 2$.
  \begin{align*}
    & \phi_0 \equiv (\forall x) \neg P(x,x)\\
    & \phi_1 \equiv (\forall x)(\exists y)P(x,y) \\
    & \phi_2 \equiv (\forall x)(\forall y)(\forall z)[P(x,y) \land P(y,z) \to P(x,z)].
  \end{align*}
\end{example}
\begin{hint}
  \begin{itemize}
  \item
    Не може да има модел само с един елемент.
  \item
    Не може да има модел само с два елемента.
  \item
    Защо не може да има модел с крайно много елементи ?
  \item
    Защо $\A = (\Nat, <)$ е модел?
  \end{itemize}
\end{hint}


\begin{example}
  За езика $\mathcal{L} = \{P,=\}$, да разгледаме следните \emph{затворени} формули:
  \begin{align*}
    & \phi_1 \equiv (\forall x)(\forall y)[\ P(x,y) \lor P(y,x) \lor x = y\ ]\\
    & \phi_2 \equiv \neg (\exists x)(\exists y)(\exists z)[\ P(x,y) \land P(y,z) \land P(z,x)\ ]\\
    & \phi_3 \equiv (\exists x)(\exists y)(\exists z)(\exists t)[\ P(x,y) \land P(y,z) \land P(z,y) \land P(t,x)\ ]\\
    & \phi_4 \equiv (\forall x)(\forall y)(\exists z)[\ P(x,z) \land P(z,y)\ ]
  \end{align*}
  Да видим кои от следните множества от формули:
  \[\Phi_1 = \{\phi_1,\phi_2\}, \Phi_2 = \{\phi_1, \phi_2, \phi_3\}, \Phi_3 = \{ \phi_1, \phi_2, \phi_4\}, \Phi_4 = \{\phi_1,\phi_2,\phi_3,\phi_4\}\]
  са изпълними и кои не.
\end{example}

\begin{example}
  Нека $\mathcal{L} = \{P, f\}$.
  \begin{align*}
    & \phi_1 \equiv (\forall x)[\ \neg P(f(x),x) \land (\exists y)[P(f(x),y)]\ ]\\
    & \phi_2 \equiv (\forall x)(\forall y)[\ P(x,y) \to (\exists z)[P(x,z) \land \neg P(z,z) \land P(z,f(x))]\ ]\\
    & \phi_3 \equiv \neg (\forall x)(\forall y)(\forall z)[\ P(x,y) \land P(y,z) \to \neg P(f(x), z)\ ]
  \end{align*}
  Да видим защо множеството $\Phi = \{\phi_1,\phi_2,\phi_3\}$ е изпълнимо.
\end{example}

\begin{example}
  За езика $\mathcal{L} = \{f,=\}$ да разгледаме затворените формули:
  \begin{align*}
    & \phi_1 \equiv (\forall x)(\forall y)(\forall z)[\ f(f(x,y),z) = f(x, f(y,z))\ ]\\
    & \phi_2 \equiv (\exists x)(\forall y)[\ f(x,y) = y \land f(y,x) = y\ ]\\
    & \phi_3 \equiv (\forall x)[\ f(x,x) = x\ ]\\
    & \phi_4 \equiv (\exists x)(\exists y)[\ \neg x = y\ ]\\
  \end{align*}
  Да видим защо множеството от формули $\Phi = \{\phi_1,\phi_2,\phi_3,\phi_4\}$ е изпълнимо.
\end{example}

\begin{example}
  За езика $\mathcal{L} = \{P,=\}$, $\sharp P = 2$, да разгледаме затворените формули:
  \begin{align*}
    & \phi_0 \equiv (\exists x)(\forall y)[x \neq y \to (\exists z)[\neg P(y,z)]]\\
    & \phi_1 \equiv (\exists x)(\forall y)P(x,y)\\
    & \phi_2 \equiv (\forall x)(\forall y)[P(x,y) \to (\exists z)[P(x,z) \land P(z,y)]]\\
    & \phi_3 \equiv (\forall x)(\exists y)P(x,y).
  \end{align*}
\end{example}
\begin{hint}
  \begin{itemize}
  \item
    $\A = (\Real^+ \cup \{0\}, \leq, =)$ като в $\phi_0$ за $x$ и $z$ изберем $0$.
  \item
    $\A = (\{a\}, P^\A,=)$, където $P = \{(a,a)\}$.
  \item
    $\A = (\{a,b\}, P^\A,=)$, където $P = \{(a,a), (a,b)\}$.
  \end{itemize}
\end{hint}


\begin{example}
  Да разгледаме $\L = \{P,Q,a,b,c\}$, където $\sharp P = \sharp Q = 2$ и $a,b,c$ са константи.
  \begin{align*}
    & \phi_0 \equiv (\exists x)[P(x,x) \land Q(x,x)]\\
    & \phi_1 \equiv (\forall x)[P(x,x) \to P(a,x)]\\
    & \phi_2 \equiv (\forall x)(\exists y)[Q(x,y) \land Q(y,b)]\\
    & \phi_3 \equiv (\exists x)[P(b,x) \land Q(c,x)]\\
    & \phi_4 \equiv Q(b,b) \land \neg P(c,c) \land \neg Q(c,c).
  \end{align*}
\end{example}
\begin{hint}
  \begin{itemize}
  \item
    Защо не е възможно да има модел само с един елемент?
  \item
    Възможно ли е да има модел само с два елемента ? Да.
    Носител $\{0,1\}$ и $P = \{(0,0)\}$ и $Q = \{(0,0),(1,0)\}$ и $a = b = 0$ и $c = 1$.
  \end{itemize}
\end{hint}

\begin{example}
  \begin{align*}
    & \phi_0 \equiv (\forall x)(\forall y)(\forall z)[f(f(x,y,a),z,a) = f(x,f(y,z,a),a)]\\
    & \phi_1 \equiv (\forall x)(\forall y)(\forall z)[f(f(x,y,b),z,b) = f(a,f(y,z,b),b)]\\
    & \phi_2 \equiv (\forall x)(\forall y)(\forall z)[f(f(x,y,a),z,b) = f(f(x,z,b),f(y,z,b),a)]\\
    & \phi_3 \equiv a = b\\
    & \phi_4 \equiv \neg(a = b)\\
    & \phi_5 \equiv (\forall x)(\forall y)(\exists z)[ f(x,y,z) \neq y].
  \end{align*}
  \begin{itemize}
  \item
    Множеството $\{\phi_0,\phi_1,\phi_2,\phi_3\}$ има модел само с един елемент.
    Също така модел е всяка структура $\A$ с непразен носител и фиксираме $a_0 \in A$ и дефинираме $a^\A = b^\A = a_0$ и $f(x,y,z) = y$.
  \item
    Да разгледаме сега множеството $\{\phi_0,\phi_1,\phi_2,\phi_4\}$.
    Тук моделът ще има поне два елемента.
    Нека $A = \{0,1\}$ и $a^\A = 0$, $b^\A = 1$ и $f^\A(x,y,z) = y$.
    Може да вземем и $f(x,y,z) = \min\{x,y,z\}$ или $f(x,y,z) = \max\{x,y,z\}$.
  \item
    Сега да разгледаме множеството $\{\phi_0,\phi_2,\phi_4, \phi_5\}$.
    Най-лесно е да разгледаме $f(x,y,z) = \max\{x,y,z\}$ и носител на структурата $\Nat$.
  \item
    Сега да разгледаме множеството $\{\phi_0,\phi_1,\phi_2,\phi_4, \phi_5\}$.
    Да разгледаме $f^\A(x,y,z) = \max{x,y} \cdot z$ и $b^\A = 0$, $a^\A = 1$.
  \end{itemize}
\end{example}


\begin{example}
  \begin{align*}
    & \phi_0 \equiv (\forall x)[f(x) \neq x \land g(x) \neq x]\\
    & \phi_1 \equiv (\exists x)[f(g(x)) = x \land g(f(x)) = x]\\
    & \phi_2 \equiv (\exists x)[f(x) = g(x)]\\
    & \phi_3 \equiv (\exists x)[g(f(x)) \neq f(g(x))]\\
    & \phi_4 \equiv (\forall y)(\exists x)[g(x) = y].
  \end{align*}
\end{example}


%%% Local Variables:
%%% mode: latex
%%% TeX-master: "../lp"
%%% End:
