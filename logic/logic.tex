\chapter{Логически основи}


\begin{itemize}
\item
  Език на съждителното смятане
\item
  Удовлетворимост на съждителна формула
\item
  Език на предикатно смятане
\item
  Определимост
\item
  Изпълнимост на предикатна формула
\item
  Тавтологии
\item
  Неизпълнимост на предикатна формула
\end{itemize}

\begin{itemize}
\item
  Големи букви - предикатни символи
\item
  Малки букви - функционални символи
\item
  Знаем арността на всеки предикатен и функционален символ.
\item
  константи - функционални символи с нулева арност.
\end{itemize}

\section{Определимост}

\begin{example}
  За езика $\L = \{P,Q\}$, където $\sharp P = 3$ и $\sharp Q = 3$,
  да разгледаме структурата $\A = (\Nat, P^\A,Q^\A)$, където
  \begin{align*}
    & P^\A(a,b,c) \stackrel{\text{деф}}{\iff} a+b = c\\
    & Q^\A(a,b,c) \stackrel{\text{деф}}{\iff} a\cdot b = c.
  \end{align*}
  Тогава следните свойства са определими:
  \begin{itemize}
  \item
    $y$ е четно
  \item
    $y$ е нечетно
  \item
    $a = 0 \iff a+b = b$. 
    Тогава можем да дефинираме $\phi_0(x) \equiv (\exists y)[P(x,y,y)]$.
    Това означава, че $\{0\} = \{a \in \Nat \mid \A \models \phi_0(a)\}$.
  \item
    Понеже $a = 1 \iff a \cdot b = b$, то
    $\phi_1(x) \equiv (\forall y)[Q(x,y,y)]$.
  \item
    $b$ е наследник на $a$ точно тогава, когато $a+1 = b$.
    Дефинираме
    $\phi_{S}(x,y) \equiv (\exists z)[\phi_1(z) \land P(x,z,y)]$.
  \item
    Тогава
    $\phi_{n+1}(x) \equiv (\exists z)[\phi_n(z) \land \phi_{S}(z,x)]$.
  \item
    $\phi_{\leq}(x,y) \equiv (\exists z)[P(x,z,y)]$.
  \item
    $\phi_{<}(x,y) \equiv (\exists z)[\neg \phi_0(z) \land P(x,z,y)]$.
  \item
    $\phi_{=}(x,y) \equiv \phi_{\leq}(x,y) \land \phi_{\leq}(y,x)$.
  \item
    $\phi_{\text{div}}(x,y) \equiv (\exists z)[Q(x,z,y)]$.
  \item
    $\phi_{\text{prime}}(x) \equiv (\exists z)[\phi_2(z) \land \phi_{\leq}(z,x) \land (\forall y)[\phi_{\text{div}}(y,x) \implies (\phi_1(y) \lor
    \phi_{=}(y,x))]]$.
    Тогава естественото число $a$ е просто точно тогава, когато $\A \models \phi_{\text{prime}}(a)$.
  \end{itemize}
\end{example}


\begin{problem}
  Да разгледаме езика $\L = \{P\}$, с $\sharp P = 3$, и структурата $\A = (\Nat, P^\A)$, където
  \[P^\A(a,b,c) \iff a + b = c+2.\]
  Докажете, че:
  \begin{enumerate}
  \item
    всеки синглетон е определим;
  \item
    релациите равенство и строго по-малко са определими.
  \end{enumerate}
\end{problem}
\begin{solution}
  Искаме формули $\phi_n(x)$, такива че
  \[\{k \in \Nat \mid \A \models \phi_n(k)\} = \{n\}.\]
  \begin{align*}
    a = 2 & \iff a + a = a + 2\\
          & \iff \A \models P(a,a,a).
  \end{align*}
  Нека $\phi_2(x) \equiv P(x,x,x)$.
  Сега имаме, че
  \begin{align*}
    a = b & \iff a + 2 = b + 2\\
          & \iff \A \models (\exists x)[ \phi_2(x) \land P(a,x,b)].
  \end{align*}
  Нека $\phi_{=}(x,y) \equiv (\exists z)[ \phi_2(z) \land P(x,z,y)]$.
  
  Сега пък можем да определим елемента $0$.
  Очевидно е, че
  \[0 + 2 = 2 \lor 1 + 1 = 2 \lor 2+0 = 2.\]
  Ние вече имаме определена двойката. Тогава
  \[\phi_0(z) \equiv (\forall x)(\forall y)[ (\neg \phi_{=}(x,y) \land P(x,y,z)) \implies (\phi_2(x) \lor \phi_2(y))].\]
  Сега вече лесно се съобразява, че
  \[\phi_{1}(z) \equiv (\exists x)[\phi_0(x) \land P(z,z,x)].\]
  Да дефинираме кога $y$ е наследник на $x$, т.е. 
  $y = x + 1 \iff y + 1 = x + 2$.
  \[\phi_{S}(x,y) \equiv (\exists z)[\phi_1(z) \land P(y,z,x)].\]
  Тогава $\phi_{n+1}(x) \equiv (\exists z)[\phi_n(z) \land \phi_S(z,x)]$.
  Също така, лесно се съобразява, че
  \begin{align*}
    a < b & \iff (\exists c)[c \geq 1\ \&\ a + c = b]\\
          & \iff (\exists c)[ c \geq 3\ \&\ a + c = b + 2]\\
          & \iff (\exists c)[ c \neq 0\ \&\ c \neq 1\ \&\ c \neq 2\ \&\ a + c = b + 2].
  \end{align*}
  Тогава
  \[\phi_{<}(x,y) \equiv (\exists z)[\neg \phi_0(z) \land \neg \phi_1(z) \land \neg \phi_2(z) \land P(x,z,y)].\]
\end{solution}

\begin{problem}
  Нека имаме език $\L = \{P\}$, с $\sharp P = 2$, и структура $\A$ за езика $\L$, където
  универсумът на $\A$ е множество от квадратчета със страна $> 0$, т.е.
  едно квадратче е множеството
  \[\{\pair{x,y} \in \Real^2 \mid 0 \leq x - a \leq n\ \&\ 0 \leq y-b \leq n\},\]
  за някои $a, b, n \in \Real$ и $n > 0$.
  Предикатът $P^\A(x,y)$ е истина точно тогава, когато квадратчетата $x$ и $y$ имат обща точка.
  Обърнете внимание, че като имаме един обект квадратче, ние нямаме начин да намерим координатите на неговите ъглови точки.
  Следните релации са определими:
  \begin{enumerate}[1)]
  \item
    Включване;
  \item
    Равенство;
  \item
    Непразно сечение (може да не е квадрат);
  \item
    Обща стена;
  \item
    Само една обща точка;
  \item
    Сечение, което е квадрат;
  \end{enumerate}
\end{problem}
\begin{solution}
  \begin{enumerate}[1)]
  \item
    $\phi_{\subseteq}(x,y) \equiv (\forall z)[P(x,z) \implies P(y,z)]$;
  \item
    $\phi_{=}(x,y) \equiv \phi_{\subseteq}(x,y) \land \phi_{\subseteq}(y,x)$;
  \item
    $\phi_{\cap}(x,y) \equiv (\exists z)[\phi_{\subseteq}(z,x) \land \phi_{\subseteq}(z,y)]$;
  \item
    Има две различни квадратчета $z_1$ и $z_2$, които се включват в $x$ и същевременно с това имат общи елементи с $y$.
    \begin{align*}
      \phi_{\text{wall}}(x,y) \equiv &  P(x,y) \land \neg \phi_{\cap}(x,y) \land\\
                                 & (\exists z_1)(\exists z_2)[\neg P(z_1,z_2) \land \phi_{\subseteq}(z_1,x) \land \phi_{\subseteq}(z_2,x) \land P(z_1,y)\land P(z_2,y)];
    \end{align*}
  \item
    Няма две различни квадратчета $z_1$ и $z_2$, които да се включват в $x$ и същевременно с това да имат общи елементи с $y$.
    \begin{align*}
      \phi_{\text{point}}(x,y) \equiv &  P(x,y) \land \neg \phi_{\cap}(x,y) \land\\
                                      & (\forall z_1)(\forall z_2)[\neg P(z_1,z_2) \land \phi_{\subseteq}(z_1,x) \land \phi_{\subseteq}(z_2,x) \to \neg(P(z_1,y)\land P(z_2,y))];
    \end{align*}
  \item
    Най-голямото сечение на $x$ и $y$ е квадрат.
    \begin{align*}
      \phi_{\text{square}}(x,y) \equiv & P(x,y) \land \\
                                       & (\exists z_1)[\phi_{\subseteq}(z_1,x) \land \phi_{\subseteq}(z_1,y) \land (\forall z_2)[\phi_{\subseteq}(z_2,x) \land \phi_{\subseteq}(z_2,y) \to \phi_{\subseteq}(z_2,z_1)]].
    \end{align*}
  \end{enumerate}
\end{solution}

\begin{problem}
  Нека имаме език $\L = \{P\}$, с $\sharp P = 2$, и структура $\A$ за езика $\L$, където
  универсумът на $\A$ е множество от кръгове с радиус $> 0$ и предикат $P^\A(x,y)$ е истина точно тогава, когато кръгчето $x$ се съдържа в кръгчето $y$.
  Следните релации са определими:
  \begin{enumerate}[1)]
  \item
    Равенство;
  \item
    Непразно сечение;
  \item
    Празно сечение;
  \item
    Вътрешен допир;
  \item
    Външен допир;
  \end{enumerate}
\end{problem}

\begin{problem}
  Да разгледаме езика $\L = \{P\}$, с $\sharp P = 3$, и структурата $\A$ над езика $\L$ с универсум $\Sigma^\star$, $\Sigma \neq \emptyset$,
  като $\A \models P(\alpha,\beta,\gamma) \iff \alpha \cdot \beta = \gamma$.
  Да се докаже, че следните множества са определими в $\A$:
  \begin{itemize}
  \item
    $\{\varepsilon\}$;
  \item
    $\{\omega \in \Sigma^\star \mid |\omega| = 1\}$;
  \item
    $\{\omega \in \Sigma^\star \mid |\omega| = 2\}$;
  \item
    $\{(\alpha,\beta) \mid \alpha \text{ е префикс на }\beta\}$;
  \item
    $\{(\alpha,\beta) \in (\Sigma^\star)^2 \mid \alpha = \beta\}$;
  \item
    $\{(\alpha,\beta) \mid \alpha \text{ е поддума на }\beta\}$;
  \end{itemize}
  Да се напише формула, която е вярна тогава и само тогава, когато азбуката $\Sigma$ е еднобуквена.
  Да се докаже, че когато $\Sigma = \{a,b\}$ не е определимо $\{aa\}$.
\end{problem}

% \begin{problem}
%   Нека $\A = (\{0,1\}^\star, =, \circ)$.
%   Докажете, че следните релации са определими:
%   \begin{itemize}
%   \item
%     $Len_n = \{\omega \mid |\omega| = n\}$;
%   \item
%     $EqLen = \{(\alpha,\beta) \mid |\alpha| = |\beta|\}$.
%   \end{itemize}
% \end{problem}


\begin{problem}
  Структурата $\A$ е с универсум множеството от всички отсечки в равнината и е за език без функционални символи и
  единствен предикатен символ $P$, който е двуместен и се интерпретира така:
  $\A \models P(t,s)$ точно тогава, когато отсечките $t$ и $s$ се пресичат.
  Забележка: Отсечките са с ненулева дължина и съдържат краищата си.
  Да се докаже, че са определими:
\begin{itemize}
\item
  $I = \{(t, s) \mid \text{отсечката $t$ се съдържа в $s$}\}$;
\item
  $E = \{(t,s) \mid \text{отсечките $t$ и $s$ са равни}\}$;
\item
  $P = \{(t,s) \mid \text{отсечките $t$ и $s$ са успоредни}\}$;
\item
  $L = \{(t, s) \mid \text{един от краищата на отсечката $t$ лежи върху $s$}\}$.
\item
  $C = \{(t, s) \mid \text{отсечките $t$ и $s$ имат общ край}\}$;
\end{itemize}
Да се докаже, че $\{(t, s) \mid \text{отсечките $t$ и $s$ са с равна дължина}\}$ не е определимо.
\end{problem}
\begin{solution}
  \begin{itemize}
  \item
    $x$ се съдържа в $y$ точно тогава, когато всяка отсечка $z$, която сече $x$ сече и $y$:
    \[\phi_{\subseteq}(x,y) \equiv (\forall z)[P(x,z) \implies P(y,z)].\]
  \item
    $x$ е равна на $y$ точно тогава, когато $x$ се съдържа в $y$ и $y$ се съдържа в $x$:
    \[\phi_{=}(x,y) \equiv \phi_{\subseteq}(x,y) \land \phi_{\subseteq}(y,x).\]
  \item
    $x$ и $y$ са успоредни точно тогава, когато не съществуват техни разширения, които се секат:
    \[\phi_{||}(x,y) \equiv (\forall z_1)(\forall z_2)[\phi_{\subseteq}(x,z_1) \land \phi_{\subseteq}(y,z_2) \implies \neg P(z_1,z_2)].\]
  \item
    един от краищата на $x$ лежи върху $y$ точно тогава, когато $x$ пресича $y$ и никоя
    отсечка, която се съдържа в $t$ не пресича $s$:
    \[\phi_{\text{point}}(x,y) \equiv P(x,y) \land (\forall z)[\phi_{\subseteq}(z,x) \implies (\phi_{=}(z,x) \lor \neg P(z,y))].\]
  \item
    $x$ и $y$ имат общ край точно тогава, когато един край на $x$ лежи върху $y$ и един край на $y$ лежи върху $x$:
    \[\phi_{\text{angle}}(x,y) \equiv \phi_{\text{point}}(x,y) \land \phi_{\text{point}}(y,x).\]
  \item
    За автоморфизъм, нека $(x,y) -> (2x,y)$.
  \end{itemize}
\end{solution}

\begin{problem}
  Ще казваме, че графът $T = (V,E)$ е дърво, ако $T$ е неориентиран свързан ацикличен граф.
  Нека $\A = (A,\subseteq,=)$ е структура с носител всички дървета $T = (V,E)$, където $V$ е крайно множество от естествени числа.
  \begin{itemize}
  \item
    Празното дърво е $T = (\emptyset,\emptyset)$ е определимо:
    \[\phi_{\text{empty}}(x) \equiv (\forall y)[x \subseteq y].\]
  \item
    Тривиалните дървета $T = (\{m\},\emptyset)$ характеризират върховете. 
    \[\phi_{\text{vertex}}(x) \equiv \neg \phi_{\text{empty}}(X) \land (\forall y)[y \subseteq x \implies (\phi_{\text{empty}}(y) \lor x = y)].\]
  \item
    $x$ е връх на реброто $y$ в дървото $z$:
    \[\phi_{\text{in}}(x,y,z) \equiv \phi_{\text{vertex}}(x) \land \phi_{\text{edge}}(z) \land x \subseteq z \land z \subseteq y.\]
  \item
    Един връх $x$ в дървото $y$ се нарича листо, ако той има степен $1$.
    \[\phi_{\text{leaf}}(x,y) \equiv \phi_{\text{vertex}}(x) \land x \subseteq y \land (\neg x = y \implies (\exists z)[\phi_{\text{in}}(x,z,y) \land (\forall z')[\phi_{\text{in}}(x,z',y) \implies z = z']]).\]
    Формулата може да се запише съкратено и така:
    \[\phi_{\text{leaf}}(x,y) \equiv \phi_{\text{vertex}}(x) \land x \subseteq y \land (\neg x = y \implies (\exists^{=1} z)[\phi_{\text{in}}(x,z,y)]).\]
  \item
    Всеки връх има степен $\leq 2$.
    \begin{align*}
      \phi_{\leq 2}(y) \equiv (\forall x)[\phi_{\text{vertex}}(x) \land x \subseteq y \implies \neg (\exists z_1)(\exists z_2)(\exists z_3)[ & \phi_{\text{in}}(x,z_1,y) \land \phi_{\text{in}}(x,z_2,y) \land \phi_{\text{in}}(x,z_3,y) \\
                                                                                                                         & \land \neg z_1 = z_2 \land \neg z_2 = z_3 \land \neg z_1 = z_3 ].
    \end{align*}
  \end{itemize}
\end{problem}


\begin{problem}
  Ще казваме, че графът $T = (V,E)$ е дърво, ако $T$ е ориентиран ацикличен граф, като съществува връх $r$ (корен), за който
  съществува път от $r$ до всеки друг връх в $V$.
  Нека $\A = (A,\subseteq,=)$ е структура с носител всички дървета $T = (V,E)$, където $V$ е крайно множество от естествени числа.
  Релацията $\subseteq$ между дървета е дефинирана като $(V_1,E_1) \subseteq (V_2,E_2) \iff V_1 \subseteq V_2\ \land E_1 \subseteq E_2$.
  \begin{itemize}
  \item
    Тривиалните дървета $T = (\{m\},\emptyset)$ характеризират върховете. 
    \[\phi_{\text{vertex}}(x) \equiv (\forall y)[y \subseteq x \implies x = y)].\]
  \item
    Дърветата от вида $T = (\{m,n\}, \{\langle m,n \rangle\})$ характеризират ребрата.
    \[\phi_{\text{edge}}(x) \equiv (\forall y)[(y \subseteq x \land \neg x = y) \implies \phi_{\text{vertex}}(y)].\]
  \item
    Върховете $x$ и $y$ са двата края на реброто $z$:
    \[\phi_{\text{connect}}(x,y,z) \equiv \phi_{\text{vertex}}(x) \land \phi_{\text{vertex}}(y) \land \phi_{\text{edge}}(z) \land x \subseteq z \land y \subseteq z \land \neg x = y.\]
  \item
    Дърветата $x$ и $y$ са съвместими, т.е. съществува дърво, което ги разширява.
    \[\phi_{\text{compat}}(x,y) \equiv (\exists z)[x \subseteq z \land y \subseteq z].\]
  \item
    Върхът $x$ е входящ за реброто $y$ точно тогава, когато може да се разшири по два различни начина с един нов връх, т.е.
    \begin{align*}
      \phi_{\text{in}}(x,y) \equiv & \phi_{\text{vertex}}(x) \land \phi_{\text{edge}}(y) \land x \subseteq y \land (\exists z)[\phi_{\text{vertex}}(z) \land \neg z \subseteq y \land \\
                                & (\exists u_1)(\exists u_2)[\neg u_1 = u_2 \land \phi_{\text{connect}}(x,z,u_1) \land \phi_{\text{connect}}(x,z,u_2) \land \\
                                & \phi_{\text{compat}}(y,u_1) \land \phi_{\text{compat}}(y,u_2)]].
    \end{align*}
  \item
    Върхът $x$ е изходящ за реброто $y$ точно тогава, когато може да се разшири по точно един начин с един нов връх, т.е.
    \begin{align*}
      \phi_{\text{out}}(x,y) \equiv & \phi_{\text{vertex}}(x) \land \phi_{\text{edge}}(y) \land x \subseteq y \land (\exists z)[\phi_{\text{vertex}}(z) \land \neg z \subseteq y \land \\
                                 & (\exists u_1)[\phi_{\text{connect}}(x,z,u_1) \land \phi_{\text{compat}}(y,u_1) \land \\
                                 & (\forall u_2)[\phi_{\text{connect}}(x,z,u_2) \land \phi_{\text{compat}}(y,u_2) \implies u_1 = u_2]]].
    \end{align*}
  \item
    Върхът $x$ е листо на дървото $y$, ако за $x$ няма изходящи ребра в $y$:
    \[\phi_{\text{leaf}}(x,y) \equiv \phi_{\text{vertex}}(x) \land x \subseteq y \land (\forall z)[\phi_{\text{edge}}(z) \land z \subseteq y \implies \neg \phi_{\text{in}}(x,z)]\]
  \item
    Върхът $x$ е корен на дървото $y$, ако за $x$ няма входящи ребра в $y$:
    \[\phi_{\text{root}}(x,y) \equiv \phi_{\text{vertex}}(x) \land x \subseteq y \land (\forall z)[\phi_{\text{edge}}(z) \land z \subseteq y \implies \neg \phi_{\text{out}}(x,z)]\]
  \item
    От върха $x$ в дървото $y$ излизат поне три ребра:
    \begin{align*}
      \phi_{\text{out}\geq 3}(x,y) \equiv \phi_{\text{vertex}}(x) \land x \subseteq y \land (\exists z_1)(\exists z_2)(\exists z_3)[ & \neg z_1 = z_2 \land \neg z_2 = z_3 \land \neg z_1 = z_3 \land \\
                                                                                                                       & z_1 \subseteq y \land z_2 \subseteq y \land z_3 \subseteq y \land\\
                                                                                                                       & \phi_{\text{out}}(x,z_1) \land \phi_{\text{out}}(x,z_2) \land \phi_{\text{out}}(x,z_3)].
    \end{align*}
  \item
    Едно дърво $x$ е двоично, ако няма връх $y$, от който да излизат повече от две ребра.
    \[\phi_{\text{bin}}(x) \equiv (\forall y)[\phi_{\text{vertex}}(y) \land y \subseteq x \implies \neg \phi_{\text{out}\geq 3}(y,x)]\]
  \end{itemize}
  
\end{problem}
  
\section{Неопределимост}

Автоморфизъм $h : \A \to \A$
\begin{itemize}
\item
  $h$ е биекция;
\item
  $P^\A(a_1,\dots,a_n) \iff P^\A(h(a_1),\dots,h(a_n))$;
\item
  $h(c^\A) = c^\A$;
\item
  $h(f^\A(a_1,\dots,a_n)) = f^\A(h(a_1),\dots,h(a_n))$.
\end{itemize}

\begin{theorem}
  $h:\A\to\A$ е автоморфизъм. Нека $\phi(\overline{x})$ е формула на езика на $\A$. Тогава
  \[\A \models \phi(a_1,\dots,a_n) \iff \A \models \phi(h(a_1),\dots,h(a_n)).\]
\end{theorem}

\begin{problem}
  Да разгледаме структурата $\A = (\mathbb{Z},+,=)$.
  Докажете, че:
  \begin{itemize}
  \item 
    $\{0\}$ е определимо.
  \item
    $\{n\}$ не е определимо, за $n \neq 0$.
  \end{itemize}
\end{problem}
\begin{hint}
  Разгледайте $h(n) = -n$.
\end{hint}

\begin{problem}
  Да разгледаме структурата $\A = (\mathbb{Q}, \cdot, =)$.
  Докажете, че:
  \begin{itemize}
  \item
    $\{0\}$, $\{1\}$ и $\{-1\}$ са определими;
  \item
    $\{q\}$ не е определимо за $q \neq 0,1,-1$;
  \end{itemize}
\end{problem}
\begin{hint}
  Разгледайте $h:\mathbb{Q} \to \mathbb{Q}$ като
  $h(0) = 0$ и $h(x) = \frac{1}{x}$.  
\end{hint}

\begin{problem}
  Да разгледаме структурата $\A = (\mathbb{N}, \cdot ,=)$.
  Докажете, че:
  \begin{itemize}
  \item
    $\{0\}$ и $\{1\}$ са определими;
  \item
    $\{n\}$ не е определимо за $n > 1$.
  \end{itemize}
\end{problem}
\begin{hint}
  Разгледайте
  \[h(x) =
    \begin{cases}
      x, & \text{ ако }x = 0 \text{ или } x = 1\\
      p^{\ell_1}_0p^{\ell_0}_1\cdots p^{\ell_k}_k, & \text{ ако }x = p^{\ell_0}_0 p^{\ell_1}_1 \cdots p^{\ell_k}_k
    \end{cases}\]

  Ясно е, че $h$ е биекция. Защо $h$ е автоморфизъм?
  С други думи, защо $h(m) \cdot h(n) = h(m \cdot n)$ ?
\end{hint}

\begin{problem}
  Да разгледаме структурата $\A = (\mathscr{P}(\Nat), \cap, =)$.  Докажете, че:
  \begin{itemize}
  \item
    $\{\emptyset\}$ и $\{\Nat\}$ са определими;
  \item
    Релациите $\subseteq$ и $=$ са определими;
  \item
    Операцията $\cup$ е определима;
  \item
    $\{A\}$ не е определимо за $A \neq \emptyset, \Nat$.
  \end{itemize}
\end{problem}
\begin{hint}
  Да фиксираме два елемента $a,b\in\Nat$.
  Да разгледаме $h:\Nat \to \Nat$ като
  \[h_{a,b}(x) =
    \begin{cases}
      b, & x = a\\
      a, & x = b\\
      x, & \text{иначе}.
    \end{cases}\]
  Тогава дефинираме $H_{a,b}:\mathscr{P}(\Nat) \to \mathscr{P}(\Nat)$ като
  $H_{a,b}(A) = \{h_{a,b}(x) \mid x \in A\}$.
  Защо $H_{a,b}$ е биекция? Защо е автормофизъм?
\end{hint}

\begin{problem}
  Да разгледаме структурата $\A = (\Nat, R)$, където $R^\A(x,y) \iff x \mid y$.
  Докажете, че:
  \begin{itemize}
  \item
    $\{0\}$ и $\{1\}$ са определими;
  \item
    Множеството $Pr$ на простите числа е определимо;
  \item
    $\{n\}$ не е определимо за $n \neq 0,1$.
  \end{itemize}
\end{problem}
\begin{hint}
  Отново разгледайте
  \[h(x) =
    \begin{cases}
      x, & \text{ ако }x = 0 \text{ или } x = 1\\
      p^{\ell_1}_0p^{\ell_0}_1\cdots p^{\ell_k}_k, & \text{ ако }x = p^{\ell_0}_0 p^{\ell_1}_1 \cdots p^{\ell_k}_k
    \end{cases}\]
  Ясно е, че $h$ е биекция. Защо $h$ е автоморфизъм?
  Нека $a = p^{\ell_0}_0 \cdot p^{\ell_1}_1 \cdots p^{\ell_k}_k$ и $b = p^{m_0}_0 \cdot p^{m_1}_1 \cdots p^{m_k}_k$. Тогава
  $a \mid b \iff \ell_0 \leq m_0\ \&\ \cdots \&\ \ell_k \leq m_k$.
\end{hint}

\begin{problem}
  Да разгледаме структурата $\A = (\Real, +, \cdot, =)$.
  Докажете, че:
  \begin{itemize}
  \item
    Релацията $\leq$ е определима.
  \item
    Всяко $\{q\}$ за $q \in \mathbb{Q}$ е определимо.
  \end{itemize}
  Намерете автоморфизмите на $\A$.
\end{problem}
\begin{hint}
  Знаем, че всяко $r \in \Real$ може да се характеризира с редица от рационални числа $(q_n)^\infty_{n=0}$, такава че
  за всяко $n$ е изпълнено:
  \[ q_n - \frac{1}{n} \leq r \leq q_n + \frac{1}{n}.\]
  Нека $h$ е автоморфизъм на $\A$. Понеже $h(q) = q$ за всяко рационално $q$, то
  $h(r)$ се характеризира със същата редица от рационални числа $(q_n)^{\infty}_{n=0}$.
  Заключаваме, че $h(r) = r$.
\end{hint}



\section{Изпълнимост}


%%% Local Variables:
%%% mode: latex
%%% TeX-master: "../lp"
%%% End:
