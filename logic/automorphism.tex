\section{Автоморфизъм на структура}

Автоморфизъм $h : \A \to \A$
\begin{itemize}
\item
  $h$ е биекция;
\item
  $P^\A(a_1,\dots,a_n) \iff P^\A(h(a_1),\dots,h(a_n))$;
\item
  $h(c^\A) = c^\A$;
\item
  $h(f^\A(a_1,\dots,a_n)) = f^\A(h(a_1),\dots,h(a_n))$.
\end{itemize}

\begin{theorem}\label{th:automorphism}
  $h:\A\to\A$ е автоморфизъм. Нека $\phi(\overline{x})$ е формула на езика на $\A$. Тогава
  \[\A \models \phi(a_1,\dots,a_n) \iff \A \models \phi(h(a_1),\dots,h(a_n)).\]
\end{theorem}

\subsection{Примери за неопределимост}

\begin{example}
  Да разгледаме структурата $\A = (\mathbb{N}, \cdot ,=)$. Да разгледаме някои от свойствата на тази структура.
  \begin{itemize}
  \item
    Както в \Example{definability:nat-plus-mult}, лесно се съобразява, че $\{0\}$ и $\{1\}$ са определими множества.
  \item
    Можем да определим кога едно число е просто:
    \[\phi_{\text{prime}}(x) \equiv \neg \phi_0(x) \land \neg \phi_1(x) \land (\forall y)(\forall z)[\ y \cdot z = x \to (\phi_1(y) \lor \phi_1(z))\ ].\]
  \item
    За разлика от преди обаче, сега ще видим, че синглетоните $\{n\}$ за $n > 1$ не са определими.
    Ще докажем това като използваме \Theorem{automorphism}.
  \end{itemize}
  
  

  Нека $p_0,p_1,p_2,\dots$ е редицата от всички прости числа, т.е. $p_0 = 2, p_1 = 3, p_2 = 5, \cdots$
  За нас ще бъде жизнено важно твърдението, че всяко положително естествено число може да се разложи по единствен начин
  като произведение от прости множители. Това означава, че всяко естествено число $x > 1$ може да се запише във вида
  $x = p^{\ell_0}_0p^{\ell_1}_1\cdots p^{\ell_k}_k$, където $\ell_k > 0$.
  Нека дефинираме функция $h:\Nat \to \Nat$ по следния начин:
  \[h(x) =
    \begin{cases}
      x, & \text{ ако }x = 0 \text{ или } x = 1\\
      p^{\ell_1}_0p^{\ell_0}_1\cdots p^{\ell_k}_k, & \text{ ако }x = p^{\ell_0}_0 p^{\ell_1}_1 \cdots p^{\ell_k}_k
    \end{cases}\]
  Лесно се съобразява, че $h$ е биекция. Защо $h$ е автоморфизъм ?
  С други думи, защо $h(x) \cdot h(y) = h(x \cdot y)$ ? Да проследим равенствата:  
  \begin{align*}
    h(x) \cdot h(y) & = h(p^{\ell_0}_0 p^{\ell_1}_1\cdots p^{\ell_k}_k) \cdot h(p^{m_0}_0 p^{m_1}_1\cdots p^{m_k}_k)\\
                    & = p^{\ell_1}_0 p^{\ell_0}_1\cdots p^{\ell_k}_k \cdot p^{m_1}_0 p^{m_0}_1\cdots p^{m_k}_k\\
                    & = p^{\ell_1+m_1}_0 p^{\ell_0+m_0}_1\cdots p^{\ell_k+m_k}_k\\
                    & = h(p^{\ell_0+m_0}_0 p^{\ell_1+m_1}_1\cdots p^{\ell_k}_k)\\
                    & = h(p^{\ell_0}_0 p^{\ell_1}_1\cdots p^{\ell_k}_k \cdot p^{m_0}_0 p^{m_1}_1\cdots p^{m_k}_k)\\
                    & = h(x \cdot y).
  \end{align*}

  Аналогично, за всяка двойка $i < j$, можем да дефинираме автоморфизма
  \[h_{i,j}(x) =
    \begin{cases}
      x, & \text{ ако }x = 0\text{ или } x = 1\\
      p^{\ell_0}_0 \cdots p_i^{\ell_j} \cdots p^{\ell_i}_j \cdots p^{\ell_k}_k, & \text{ ако }x = p^{\ell_0}_0 \cdots p^{\ell_i}_i \cdots p^{\ell_j}_j \cdots p^{\ell_k}_k
    \end{cases}
  \]
  
  Сега лесно се съобразява, че, например, $\{12\}$ не е определимо.
  За да бъде определимо, трябва $\{12\} = \{h_{0,1}(12)\}$, т.е. $h_{0,1}(12) = 12$, но
  $h_{0,1}(2^2 \cdot 3^1) = 2^1 \cdot 3^2 \neq 12$.

\end{example}


\begin{example}
  За произволни $a,b \in \Real$, ще казваме, че
  \[\ell_{a,b} \df \{\pair{\lambda a, \lambda b} \in \Real^2 \mid \lambda \in \Real^+\}\]
  е \emph{лъч} с начало $(0,0)$. Тук $\Real^+ = \Real \cap [0,+\infty)$.
  Лъчът $\ell_{0,0}$ ще наричаме \emph{тривиален}.

  За едно непразно подмножество $C$ на $\Real^2$ ще казваме, че е \emph{конус}, ако удовлетворява следното свойство:
  \begin{equation}
    \label{eq:1}
    \pair{a,b} \in C \iff \ell_{a,b} \subseteq C.
  \end{equation}
  Обърнете внимание, че всеки лъч е частен случай на конус.
  Нека $\texttt{Cone}$ да бъде съвкупността от всички конуси. Да разгледаме структурата $\A = (\mathcal{P}(\Real^2), Cone^\A, Cut^\A)$, където
  \begin{align*}
    & Cone^A(c) \stackrel{\text{деф}}{\iff} c \in \texttt{Cone}\\
    & Cut^\A(c_1,c_2) \stackrel{\text{деф}}{\iff} (c_1 \cap c_2) \setminus \ell_{0,0} \neq \emptyset.
  \end{align*}
  \begin{itemize}
  \item
    Можем да определим тривиалния лъч $\ell_{0,0}$ в $\A$ по следния начин:
    \[\phi_{\text{trivial}}(x) \equiv Cone(x) \land (\forall y)[\ \neg Cut(x,y)\ ].\]
    Можем да определим също така множеството $\emptyset$ и множеството $\Real$ в $\A$ по следния начин:
    \begin{align*}
      & \phi_{\emptyset}(x) \equiv \neg Cone(x) \land (\forall y)[\ \neg Cut(x,y)\ ].\\
      & \phi_{\Real}(x) \equiv (\forall y)[\ (\neg \phi_{\text{trivial}}(y) \land \neg \phi_{\emptyset}(y)) \to Cut(x,y)\ ].
    \end{align*}
  \item
    Можем да определим релацията \emph{,,включване''} на един конус $x$ в друг конус $y$:
    \[\phi_{\subseteq}(x,y) \equiv Cone(x) \land Cone(y) \land (\forall z)[Cut(x,z) \to Cut(y,z)].\]
    Автоматично вече имаме определение и на релацията \emph{,,равенство''} на конусите $x$ и $y$:
    \[\phi_{=}(x,y) \equiv \phi_{\subseteq}(x,y) \land \phi_{\subseteq}(y,x).\]
  \item
    Можем също така да определим кога едно множество представлява лъч така:
    \[\phi_{\text{line}}(x) \equiv Cone(x) \land (\forall y)[\ (\phi_{\subseteq}(y,x) \land \neg \phi_{\text{trivial}}(y)) \to \phi_{=}(y,x)\ ].\]
    С други думи, множеството $\texttt{Line}$ от всички лъчи е определимо.
  \item
    Сега ще видим, макар и $\texttt{Line}$ и $\{\ell_{0,0}\}$ да са определими,
    то за никоя двойка $\pair{a,b} \neq \pair{0,0}$ не е вярно, че синглетонът $\{\ell_{a,b}\}$ е определим.
    За целта първо дефинираме функцията $h:\Real^2 \to \Real^2$ като:
    \[h(\pair{a,b}) \df \pair{-a,-b}.\]
    Очевидно е, че $h(\pair{-a,-b}) = \pair{a,b}$.
    Лесно се съобразява, че $h$ е биекция.
    Сега нека $H:\mathcal{P}(\Real^2) \to \mathcal{P}(\Real^2)$ дефинираме така:
    \[H(A) \df \{h(\pair{a,b}) \mid \pair{a,b} \in A\}.\]
    Отново лесно се съобразява, че $H$ е биекция.
    Интересното е да видим защо $H$ задава автоморфизъм на $\A$.
    Първо да отбележим, че имаме свойствата:
    \begin{align}
      & A \subseteq B \iff H(A) \subseteq H(B) \label{eq:2}\\
      & H(\ell_{a,b}) = \ell_{-a,-b} \label{eq:3}\\
      & H(\ell_{-a,-b}) = \ell_{a,b} \label{eq:4}
    \end{align}
    Нека $C$ е конус. Да видим защо $H(C)$ е конус, т.е. да проверим Свойство~(\ref{eq:1}).
    Да проследим еквивалентностите:
    \begin{align*}
      \pair{a,b} \in H(C) & \iff h(\pair{-a,-b}) \in H(C)\\
                          & \iff \pair{-a,-b} \in C & \comment\text{деф. на }H\\
                          & \iff \ell_{-a,-b} \subseteq C & \comment\text{от Свойство~(\ref{eq:1})}\\
                          & \iff H(\ell_{-a,-b}) \subseteq H(C) & \comment\text{от Свойство~(\ref{eq:2}})\\
                          & \iff \ell_{a,b} \subseteq H(C). & \comment\text{от Свойство~(\ref{eq:4}})
    \end{align*}
    Аналогично, нека $H(C)$ е конус. Да видим защо $C$ е конус.
    \begin{align*}
      \pair{a,b} \in C & \iff h(\pair{a,b}) \in H(C)\\
                       & \iff \pair{-a,-b} \in H(C) & \comment\text{деф. на }H\\
                       & \iff \ell_{-a,-b} \subseteq H(C) & \comment\text{от Свойство~(\ref{eq:1})}\\
                       & \iff H(\ell_{a,b}) \subseteq H(C) & \comment\text{от Свойство~(\ref{eq:3}})\\
                       & \iff \ell_{a,b} \subseteq C. & \comment\text{от Свойство~(\ref{eq:2}})
    \end{align*}

    Сега трябва да се справим с втория предикат в структурата $\A$.
    Тук лесно се проверява, че
    \[(A\cap B)\setminus \ell_{0,0} \neq \emptyset \iff (H(A) \cap H(B)) \setminus \ell_{0,0} \neq \emptyset,\]
    защото имаме за произволно $A \subseteq \Real^2$, че:
    \[\ell_{0,0} \in A \iff \ell_{0,0} \in H(A).\]
    Така заключаваме, че $H$ е автоморфизъм на $\A$.
    
    Накрая, ако допуснем, че $\{\ell_{a,b}\}$ е определимо, то бихме имали, че $\{\ell_{a,b}\} = \{H(\ell_{a,b})\}$.
    Това е възможно само за $\pair{a,b} = \pair{0,0}$.
  \item
    За произволни $a,b \in \Real$, ще казваме, че
    \[\pi_{a,b} \df \{ \pair{\lambda a, \lambda b} \in \Real^2 \mid \lambda \in \Real\}\]
    е \emph{права}.
    Сега вече лесно можем да докажем, че за всяка двойка $\pair{a,b} \neq \pair{0,0}$,
    синглетонът $\{\pi_{a,b}\}$ не е определим.
    Обърнете внимание, че $\pi_{a,b} = \pi_{-a,-b} = H(\pi_{a,b})$, т.е. не може да използваме предишния автоморфизъм
    за да докажем неопределимост на синлетоните $\{\pi_{a,b}\}$.
    Разгледайте $G:\mathcal{P}(\Real^2) \to \mathcal{P}(\Real^2)$ като
    \[G(A) \df \{\pair{a,-b} \mid \pair{a,b} \in A\}.\]
    Аналогично можем да докажем, че $G$ е автоморфизъм.
    Тогава лесно се съобразява, че $G(\pi_{a,b}) \neq \pi_{a,b}$.
  \end{itemize}
\end{example}


\begin{problem}
  Да разгледаме структурата $\A = (\mathbb{Z},+,=)$.
  Докажете, че:
  \begin{itemize}
  \item 
    $\{0\}$ е определимо.
  \item
    $\{n\}$ не е определимо, за $n \neq 0$.
  \end{itemize}
\end{problem}
\begin{hint}
  Разгледайте $h(n) = -n$.
\end{hint}

\begin{problem}
  Да разгледаме структурата $\A = (\mathbb{Q}, \cdot, =)$.
  Докажете, че:
  \begin{itemize}
  \item
    $\{0\}$, $\{1\}$ и $\{-1\}$ са определими;
  \item
    $\{q\}$ не е определимо за $q \neq 0,1,-1$;
  \end{itemize}
\end{problem}
\begin{hint}
  Разгледайте $h:\mathbb{Q} \to \mathbb{Q}$ като
  $h(0) = 0$ и $h(x) = \frac{1}{x}$.  
\end{hint}


\begin{problem}
  Да разгледаме структурата $\A = (\mathscr{P}(\Nat), \cap, =)$.  Докажете, че:
  \begin{itemize}
  \item
    $\{\emptyset\}$ и $\{\Nat\}$ са определими;
  \item
    Релациите $\subseteq$ и $=$ са определими;
  \item
    Операцията $\cup$ е определима;
  \item
    $\{A\}$ не е определимо за множества $A$ различни от $\emptyset$ и $\Nat$.
  \end{itemize}
\end{problem}
\begin{hint}
  Да фиксираме два различни елемента $a,b\in\Nat$.
  Да разгледаме $h:\Nat \to \Nat$ като
  \[h_{a,b}(x) =
    \begin{cases}
      b, & x = a\\
      a, & x = b\\
      x, & \text{иначе}.
    \end{cases}\]
  Тогава дефинираме $H_{a,b}:\mathscr{P}(\Nat) \to \mathscr{P}(\Nat)$ като
  $H_{a,b}(A) = \{h_{a,b}(x) \mid x \in A\}$.
  Защо $H_{a,b}$ е биекция? Защо е автормофизъм?
\end{hint}

\begin{problem}
  Да разгледаме структурата $\A = (\Nat, R)$, където $R^\A(x,y) \iff x \mid y$.
  Докажете, че:
  \begin{itemize}
  \item
    $\{0\}$ и $\{1\}$ са определими;
  \item
    Множеството $\texttt{Prime}$ на простите числа е определимо;
  \item
    $\{n\}$ не е определимо за $n \neq 0,1$.
  \end{itemize}
\end{problem}
\begin{hint}
  Отново разгледайте
  \[h(x) =
    \begin{cases}
      x, & \text{ ако }x = 0 \text{ или } x = 1\\
      p^{\ell_1}_0p^{\ell_0}_1\cdots p^{\ell_k}_k, & \text{ ако }x = p^{\ell_0}_0 p^{\ell_1}_1 \cdots p^{\ell_k}_k
    \end{cases}\]
  Ясно е, че $h$ е биекция. Защо $h$ е автоморфизъм?
  Нека $a = p^{\ell_0}_0 \cdot p^{\ell_1}_1 \cdots p^{\ell_k}_k$ и $b = p^{m_0}_0 \cdot p^{m_1}_1 \cdots p^{m_k}_k$. Тогава
  $a \mid b \iff \ell_0 \leq m_0\ \&\ \cdots \&\ \ell_k \leq m_k$.
\end{hint}

\begin{problem}
  Да разгледаме структурата $\A = (\Real, +, \cdot, =)$.
  Докажете, че:
  \begin{itemize}
  \item
    Релацията $\leq$ е определима.
  \item
    Всяко $\{q\}$ за $q \in \mathbb{Q}$ е определимо.
  \end{itemize}
  Намерете автоморфизмите на $\A$.
\end{problem}
\begin{hint}
  Знаем, че всяко $r \in \Real$ може да се характеризира с редица от рационални числа $(q_n)^\infty_{n=0}$, такава че
  за всяко $n$ е изпълнено:
  \[ q_n - \frac{1}{n} \leq r \leq q_n + \frac{1}{n}.\]
  Нека $h$ е автоморфизъм на $\A$. Понеже $h(q) = q$ за всяко рационално $q$, то
  $h(r)$ се характеризира със същата редица от рационални числа $(q_n)^{\infty}_{n=0}$.
  Заключаваме, че $h(r) = r$.
\end{hint}



%%% Local Variables:
%%% mode: latex
%%% TeX-master: "../lp"
%%% End:
